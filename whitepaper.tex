\documentclass[twocolumn]{article}

\begin{document}
\frenchspacing

\title{Sia: Simple Decentralized Storage}

\author{
{\rm David Vorick}\\
Nebulous Inc.\\
david@nebulouslabs.com
\and
{\rm Luke Champine}\\
Nebulous Inc.\\
luke@nebulouslabs.com
}

\maketitle

\subsection*{Abstract}
Sia is a blockchain-based cryptosystem that enables decentralized file storage.
This paper serves as both a high level overview and as a specification.

\section{Introduction}
Decentralized storage is good for the health of the Internet.

Sia is an altcoin that will be merge-mined with Bitcoin.

Altcoin is needed for enforcing contracts that require storage proofs.

\section{Block Structure}
Block Header:
\begin{enumerate}
	\item Version
	\item Previous Block Hash
	\item Timestamp
	\item Block Hash
\end{enumerate}

\[
Block Hash = H(Nonce || Transactions)
\]

\section{Transactions}
Each transaction:
\begin{enumerate}
	\item Version
	\item Type
	\item Input List
	\item Output List
	\item Type Field
	\item Transaction Signatures
\end{enumerate}

\subsection{Type}
There are 3 types of transactions.
\begin{enumerate}
	\item Money Transfer
	\item Host Announcement
	\item File Contract
	\item Proof of Storage
\end{enumerate}

The type field does different things for each type of transaction.

\subsection{Inputs}
There are up to 16 inputs.

Each Input:
\begin{enumerate}
	\item Previous Output ID
	\item Previous Output Spend Conditions
\end{enumerate}

\subsection{Outputs}
There are up to 16 outputs.

Each Output:
\begin{enumerate}
	\item Hash of Output Spend Conditions.
\end{enumerate}

Outputs are identified by id.
The output id is derived from the transaction that the output appeared in.
The output id is derived by taking H(first valid input id || output index).
The input id is valid if the input spends a previously existing output or is a block subsidy.

\subsection{Spend Conditions}
\begin{enumerate}
	\item Timelock
	\item Number of Signatures Required
	\item Number of Valid Public Keys
	\item Each Valid Public Key
\end{enumerate}

The timelock prevents an output from being spent until a certain time.

The number of signatures and public keys creates an m-of-n multisig, where m and n can be as low as 1 and as high as 32.

\subsection{Signatures}
There can be a maximum of 32 * 16 signatures, 32 signatures per input for 16 total inputs.

Each Signature:
\begin{enumerate}
	\item Timelock
	\item Input ID
	\item Signature Array
	\item Actual Signature
\end{enumerate}

The timelock prevents a signature from being valid until a certain time.
The signature array indicates what is being signed.
It is a 32 bit array, one bit for each input and one bit for each output.
This prevents certain inputs and outputs within the transaction from being changed without invalidating the signature.

The input id must be an input in the transaction.

The signature covers:
\begin{enumerate}
	\item The Transaction Version
	\item The Transaction Type
	\item The Transaction Type Field
	\item The Signature Timelock
	\item The Signature Input ID
	\item The Signature Signature Array
	\item Every input and output specified by the signature array
\end{enumerate}

\section{Money Transfers}
Money transfers are the generic transaction.
They have no type field.

\section{Host Announcement}
Host announcements have a type field up to 1kb that can be used for arbitrary data.

This is to tell potential clients about themselves, what files they store (which can be viewed in the file contracts), and any policy information.
It's completely up to the hosts and clients to figure out how to use this 1kb in a useful way.

Most host announcements will likely consist of only a transaction fee, though some may perhaps also create a timelocked output as a demonstration of their committment to being a host.

\section{File Contracts}
File contracts are an agreement between a host and an uploader that the host will store a file.

The inputs on the transaction are placed into a 'reward pool', which are conditionally sent to various outputs.

The type field for a file contract contains:
\begin{enumerate}
	\item A merkle root hash of the file being stored.
	\item The size of the underlying file.
	\item A public key for the host to do file proofs against.
	\item An up to 1kb field for arbitrary data.
	\item A file proof frequency, maximum of 1 every block, minimum of 1 every 2016 blocks.
	\item A reward to be split between some outputs if a file proof is found.
	\item A reward to be split between some outputs if a file proof is not found.
	\item A reward to be split between some outputs if a file proof is not found for N consecutive cylces.
	\item A field indicating how the first reward scales. (??)
	\item A field indicating how the second reward scales. (??)
	\item A field indicating how the third reward scales. (??)
\end{enumerate}

The frequency gives a window in which the host can submit a storage proof.
If the frequency is 1 every 20 blocks, then for each non-overlapping set of 20 blocks, a host can put a proof-of-storage in \textbf{any} of them.

If the proof of storage appears, then some of the reward pool (which starts as the sum of all the inputs) is moved into the R1 outputs.
Presumably, the R1 outputs all belong to the machines hosting the file.

If the proof of storage does not appear in a certain window, then some of the reward pool is moved to the R2 outputs.
Presumably, these outputs belong to the file uploader, and this acts as a refund.

If the proof of storage does not appear for N consecutive windows, then some of the reward pool is moved to the R3 outputs.
Presumably, this is a bigger reward, and it goes to the uploader, and acts as a larger refund, or even a host penalty.

The scaling factors are currently underspecified, but will likely be a set of flags that enable contracts to feature increasingly painful penalities to the hosts as more storage proofs are missed.

The file id is defined by the hash of the transaction announcing the file.

The outputs of the file are not spendable until the reward pool has been fully distributed.

\section{Proof of Storage}
Each proof of storage transaction targets a specific file.
A storage proof does not need to have any inputs or outputs.

A storage proof has:
\begin{enumerate}
	\item A file id
	\item A merkle proof proving the active index for the file.
	\item A signature of the file id and the merkle proof concatenated, using the public key found in the file contract transaction.
\end{enumerate}

The merkle proof is performed on a random leaf in the file being stored.
This 'random' leaf is chosen deterministically using the hash of the block immediately before the proof window as a seed.
This random number could be manipulated using a block witholding attack or a 51\% attack, but where an attacker only has one third hashing power, the chances of pulling off a worthwhile attack are slim.
Especially if there are many proofs that need to be performed, this method of generating random numbers is strong enough for most files because storage is cheap.

It should be noted that there is another attack, the "closed window" attack.
If a miner refuses to add a storage proof transaction to a block for enough blocks in a row, then the file host will not be able to complete the proof of storage even though they are online and have the file.
Once again this is not a very likely attack because there are many miners, and hosts are free to refuse to hold any file where the storage window is less than 20 blocks.

\section{Being a Host}
As a host, you would make an announcement using a host announcement and follow some standard for organizing the announcement.
This would contain information about how to contact you, your typical price and penaltys, and some cryptographic indication of which files you are already storing.

Clients would then contact you, requesting you hold onto files.
You agree to hold onto files, but only after they have given you the whole file.

You use microchannel payments to charge the clients for downloading the file.

As a host, you are protected against files you don't want because you don't have to agree to store anything.

\section{Being an Uploader}
You break the file into a bunch of erasure coded pieces, using M of N erasure coding.
You encrypt the N pieces individually and upload them each to a different host.

You find hosts by scanning the host announcements and having some protocol for reading them.
You create a pool of hosts that fit your uploading criteria, and contact them at random weighted according to how much you trust them.
The randomness protects you from malicious hosts and network instability.

You upload the file to each of the hosts and they post a file contract agreeing to store them.
You check in on the file every few months and do any repairs needed if hosts go offline.

You use microchannel payments to negotiate downloads.
You are protected against hosts holding your data hostage because you can just go to other hosts who are not holding it hostage.
When data is available in many places, the places individually do not have much power over you.

\section{Tweaks}
6 minute block time.

Difficulty adjustments every block. (note - dishonest difficulty attack - ask wizards if there is a fix)
Max adjustments of 1.0005x per block, corresponding to a max change of 5.3x every 2 weeks, compared to Bitcoin's clamp at 4x every 2 weeks.

Block subsidy adjustments every block.
Inflation initially be very high, around 25\% in year two.
By year 4, inflation will taper to 5\% anually, and it will stay at that rate permenently.

\section{Conclusion}

\onecolumn
\begin{bibliography}{2}

\bibitem{}

\end{bibliography}

\end{document}
