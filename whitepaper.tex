\documentclass[twocolumn]{article}

\begin{document}
\frenchspacing

\title{Sia: Simple Decentralized Storage}

\author{
{\rm David Vorick}\\
Nebulous Inc.\\
david@nebulouslabs.com
\and
{\rm Luke Champine}\\
Nebulous Inc.\\
luke@nebulouslabs.com
}

\maketitle

\subsection*{Abstract}
Sia is a blockchain-based cryptosystem that enables decentralized file storage.
This paper serves as both a high level overview and as a specification.

\section{Introduction}
Decentralized storage is good for the health of the Internet.

Sia is an altcoin that will be merge-mined with Bitcoin.

Altcoin is needed for enforcing contracts that require storage proofs.

\section{Block Structure}
Block Header:
\begin{enumerate}
	\item Version
	\item Previous Block Hash
	\item Timestamp
	\item Block Hash
\end{enumerate}

\[
Block Hash = H(Nonce || Transactions)
\]

\section{Transactions}
Each transaction:
\begin{enumerate}
	\item Version
	\item Type
	\item Input List
	\item Output List
	\item Type Field
	\item Transaction Signatures
\end{enumerate}

\subsection{Type}
There are 3 types of transactions.
\begin{enumerate}
	\item Money Transfer
	\item Host Announcement
	\item File Contract
\end{enumerate}

The type field does different things for each type of transaction.

\subsection{Inputs}
There are up to 16 inputs.

Each Input:
\begin{enumerate}
	\item Previous Output ID
	\item Previous Output Spend Conditions
\end{enumerate}

\subsection{Outputs}
There are up to 16 outputs.

Each Output:
\begin{enumerate}
	\item Hash of Output Spend Conditions.
\end{enumerate}

Outputs are identified by id.
The output id is derived from the transaction that the output appeared in.
The output id is derived by taking H(first valid input id || output index).
The input id is valid if the input spends a previously existing output or is a block subsidy.

\subsection{Spend Conditions}
\begin{enumerate}
	\item Timelock
	\item Number of Signatures Required
	\item Number of Valid Public Keys
	\item Each Valid Public Key
\end{enumerate}

The timelock prevents an output from being spent until a certain time.

The number of signatures and public keys creates an m-of-n multisig, where m and n can be as low as 1 and as high as 32.

\subsection{Signatures}
There can be a maximum of 32 * 16 signatures, 32 signatures per input for 16 total inputs.

Each Signature:
\begin{enumerate}
	\item Timelock
	\item Input ID
	\item Signature Array
	\item Actual Signature
\end{enumerate}

The timelock prevents a signature from being valid until a certain time.
The signature array indicates what is being signed.
It is a 32 bit array, one bit for each input and one bit for each output.
This prevents certain inputs and outputs within the transaction from being changed without invalidating the signature.

The input id must be an input in the transaction.

The signature covers:
\begin{enumerate}
	\item The Transaction Version
	\item The Transaction Type
	\item The Transaction Type Field
	\item The Signature Timelock
	\item The Signature Input ID
	\item The Signature Signature Array
	\item Every input and output specified by the signature array
\end{enumerate}

\section{Money Transfers}
Money transfers are the generic transaction.
They have no type field.

\section{Host Announcement}
Host announcements have a 1kb type field that arbitrary data can be placed in.

This is to tell potential clients about themselves, what files they store (which can be viewed in the file contracts), and any policy information.
It's completely up to the hosts and clients to figure out how to use this 1kb in a useful way.

Most host announcements will likely consist of only a transaction fee, though some may perhaps also create a timelocked output as a demonstration of their committment to being a host.

\section{File Contracts}

\section{Tweaks}
6 minute block time.

Difficulty adjustments every block. (note - dishonest difficulty attack - ask wizards if there is a fix)
Max adjustments of 1.0005x per block, corresponding to a max change of 5.3x every 2 weeks, compared to Bitcoin's clamp at 4x every 2 weeks.

Block subsidy adjustments every block.
Inflation initially be very high, around 25\% in year two.
By year 4, inflation will taper to 5\% anually, and it will stay at that rate permenently.

\section{Conclusion}

\onecolumn
\begin{bibliography}{2}

\bibitem{}

\end{bibliography}

\end{document}
